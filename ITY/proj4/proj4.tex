\documentclass[11pt, a4paper, titlepage]{article}
\usepackage[czech]{babel}
\usepackage[utf8]{inputenc}
\usepackage[IL2]{fontenc}
\usepackage[left=2cm,top=3cm,text={17cm,24cm}]{geometry}
\usepackage{times}
\usepackage{hyperref}
\bibliographystyle{czechiso}

\begin{document}
\begin{titlepage}
\begin{center}
{\Huge\textsc{
Vysoké učení technické v~Brně\\[0.4em]}}
{\huge\textsc{
Fakulta informačních technologií}}
\vspace{\stretch{0.382}}

{\LARGE
Typografie a publikování\,--\,4. projekt\\}
\Huge{Bibliografické citace}
\vspace{\stretch{0.618}}

{\Large
\today
\hfill
Petr Bartoš
}
\end{center}
\end{titlepage}

\section{Typografie}
\subsection{Co je to typografie}
Typografii můžeme dle některých autorů chápat jako formu umění\,--\,autor je pomocí ní schopen vyvolat různé emoce či předat zprávu, čehož by použitím „běžného“ textu nebyl schopen dosáhnout \cite{bringhurst2019}. V~dnešní době typografií rozumíme disciplínu zabývající se výběrem, použitím a sazbou písma. 

Typografii můžeme dále rozdělit na makrotypografii a mikrotypografii \cite{TypografieaText}. Zatímco makrotypografie se zabývá textem jako celkem, tedy věcmi, jako jsou například rozmístění odstavců, formát stránky či výběr velikosti okrajů, mikrotypografie se zabývá písmem a jeho použitím, tedy interpunkcí, mezerami či dokonce návrhem a tvorbou písma \cite{TypographicalCues}.  

\subsection{Historie}
Vznik písma je nejčastěji připisován Sumerům, ač se archeologové podle posledních objevů domnívají, že existovaly ještě starší civilizace, které písmo používaly před nimi \cite{Archaeology}. S~časem se posouvala i komplexnost písma, což můžeme pozorovat například na egyptských hieroglyfech, které pomocí tzv. piktogramů reprezentovaly jednotlivé objekty \cite{VACLAVKOVA2009}.

V~době středověku se těšily velkému úspěchu ilustrace. Velkým milníkem byl však vznik knihtisku, který tehdejší práci s~písmem přiblížil k~typografii, jak ji známe dnes \cite{DVORAKOVA2012}. Poté následovaly úpravy písma samotného, zprvu šlo hlavně o~ušetření místa na papíru, v~pozdějších letech se vývoj ubíral směrem čitelnosti a zdobnosti \cite{haley2012}. Čeští čtenáři mohli dlouhá léta sledovat trendy v~typografii po celém světě v~časopisu Typografia \cite{Typografia}. 

\subsection{Typografie a moderní doba}
S~rozmachem moderních technologií a internetu se však přístup k~informacím stal ještě dostupnější než kdykoliv předtím. Je tedy nesmírně jednoduché nejen zjistit, jak s~textem pracovat, dokonce i tvorba vlastního fontu je otázka „nakreslení“ znaků a jejich spojení do jednoho souboru, který pak může použít kdokoliv s~připojením k~internetu \cite{FontGuide}.

V~dnešní uspěchané době hraje typografie důležitou roli\,--\,je totiž to, co udrží čtenářovu pozornost. Sdělení může být sebelepší, pokud je ale prezentováno špatným způsobem, moderní člověk, který je zvyklý textem „proletět“, mu svou pozornost věnovat nebude \cite{TypographyPsychological}.

\newpage
\bibliography{proj4}

\end{document}